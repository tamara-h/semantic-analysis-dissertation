\section{Abstract}

The semantic analysis investigation produced by this project explores structures in which emotions can be represented in a computable format, primarily using a Valence-Arousal-Dominance structure, and optimising sentiment prediction tools using lexicon-based and machine learning methods. Existing semantic analysis tools usually only classify whether input text is positive or negative, and this project aims to explore other ways text can be classified.
Natural Language Processing tools are applied to text-based datasets, and various investigations are carried out to explore the best way to utilise the data for a prediction tool.

This project produces a machine learning model that takes an input sentence and returns Positive/Neutral/Negative classes for the Valence, Arousal and Dominance for the emotion behind the sentence. This model is then applied to a web application that uses this returned data to relate the text to music. 

The findings set out by this project imply that using a more complex structure to represent emotion can lead to a better understanding of input text, and shows that we can apply existing sentiment prediction methods to this structure to obtain an effective model.
\\
\textit{Keywords: Semantic Analysis, Machine Learning, Emotion Prediction, Text Analysis}